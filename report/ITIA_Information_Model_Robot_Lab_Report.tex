\documentclass{article}
\usepackage[utf8]{inputenc}
\usepackage{amsmath}
\usepackage{graphicx}
\usepackage{float}
\usepackage{amssymb}
\usepackage{listings}
\usepackage{pgfplots}
\usepackage{siunitx}
\usepackage{tabularx}
\usepackage{ragged2e}
\usepackage{multirow}
\usepackage{a4wide}
\usepackage{hyperref}

\pgfplotsset{compat=1.15}
\title{ITIA\\Information Model Robot\\Lab Report}
\author{Stefan Adelmann (01633044)\\Hannes Brantner (01614466)}
\begin{document}
\maketitle{}
The information model of the Robot Station consists of a OPC UA Gateway providing methods that interact with the Robot Controller. 
The OPC UA Gateway is provided by a Raspberry Pi, that can be accessed via the information given in \autoref{tab:gate}.
\begin{table}[H]
	\centering
	\setlength\extrarowheight{2pt}
	\begin{tabular}[h]{|p{1.2cm}p{4.5cm}|}
		\hline
		\multicolumn{2}{|c|}{\bf OPC UA Gateway}\\
		\hline\hline
		Device: & RaspberryPi 3\\
		\hline
		ID: & BCM2835 (a02082)\\
		\hline
		MAC: & b8:27:eb:09:db:ca\\
		\hline
		IP: & 192.168.162.84/25\\
		\hline
		Port: & 4840\\
		\hline
	\end{tabular} 
	\caption{OPC UA Gateway information}
	\label{tab:gate}
\end{table}
The robot controller itself is accessed by the OPC UA gateway via the connection information given in \autoref{tab:robCtrl}.
\begin{table}[H]
	\centering
	\setlength\extrarowheight{2pt}
	\begin{tabular}[h]{|p{1.2cm}p{4.5cm}|}
		\hline
		\multicolumn{2}{|c|}{\bf Robot Controller}\\
		\hline\hline
		Device: & Robot Controller\\
		\hline
		ID: & CR750-D\\
		\hline
		MAC: & 38:e0:8e:9e:89:8d\\
		\hline
		IP: & 192.168.162.82/25\\
		\hline
		PORT: & 10003\\
		\hline
	\end{tabular} 
	\caption{Robot Controller information}
	\label{tab:robCtrl}
\end{table}
Together these two components form a system in which all aspects needed for the implementation of the Robot Station can be controlled via the OPC UA interface. In order to control aspects like the spring storage or retrieve information of the built in sensors, the I/O mapping presented in the following section has to be used.
\newpage
\section{I/O Mapping}
The I/O mapping is divided into the two groups of sensors and actuators. It is to note that the movement of the robot arm as well as the movement of the multi-function gripper is omitted from these tables as they are accessed via proprietary OPC UA methods. 
\begin{table}
	\setlength\extrarowheight{4pt}
	\small
	\begin{tabularx}{\textwidth}{|p{1cm}|X|}
		\hline
		\multicolumn{2}{|c|}{\bf \color{black} \large Sensors}\\
		\hline\hline
		\bf Index & \bf Description\\
		\hline\hline
		1 & Robot handling module - workpiece alignment\\
		\hline
		2 & Robot handling module - workpiece in pick-up position\\
		\hline
		3 & Control panel - Start (make contact)\\
		\hline
		4 & Control panel - Stop (normally closed) \\
		\hline
		5 & Control panel - Reset (no contact)\\
		\hline
		7 & Control panel - COM bridge (I7)\\
		\hline
		8 & Robot assembly module (spring magazine) - slider retracted\\
		\hline
		9 & Robot assembly module (spring magazine) - slide extended\\
		\hline
		10 & Module robot assembly (spring magazine) - spring available \\
		\hline
		12 & Robot assembly module (lid magazine) - slide retracted\\
		\hline
		13 & Robot assembly module (lid magazine) - slide extended\\
		\hline
		15 & Module robot assembly (lid magazine) - lid on tray\\
		\hline
		900 & Robot module (hand) - part not black\\
		\hline
	\end{tabularx}

	\setlength\extrarowheight{4pt}
	\small
	\begin{tabularx}{\textwidth}{|p{1cm}|X|}
		\hline
		\multicolumn{2}{|c|}{\bf \color{black} \large Actuators}\\
		\hline\hline
		\bf Index & \bf Description\\
		\hline\hline
		0 & Control panel - Start (LED)\\
		\hline
		1 & Control panel - Reset (LED)\\
		\hline
		2 & Control panel - Q1 (LED)\\
		\hline
		3 & Control panel - Q2 (LED)\\
		\hline
		4 & Control panel - COM bridge (Q4)\\
		\hline
		8 & Robot assembly module (spring magazine) - slide out\\
		\hline
		12 & Robot assembly module (lid magazine) - slide out\\
		\hline
	\end{tabularx}
	\caption{I/O mapping}
	\label{tab:io}
\end{table}
This distinction is not only present in the OPC UA model but also in the native programming language of the Robot Controller.

\section{OPC UA Methods}
\autoref{tab:methods} presents all custom methods that are provided by the OPC UA gateway using the generated information model. They are a collection of direct control elements needed to implement certain robot behavior as well as management elements needed to keep the robot running. The later primarily involves error handling as error states need to be reset in order to operate.
\begin{table}[H]
	\setlength\extrarowheight{4pt}
	\small
	\begin{tabularx}{\textwidth}{|p{5cm}|p{4.5cm}|X|X|}
		\hline
		\multicolumn{4}{|c|}{\bf \color{black} \large OPC UA Methods}\\
		\hline\hline
		\bf Method &\bf Description & \bf ObjectId & \bf MethodId\\
		\hline\hline
		ReadInput(Index) & Returns the \texttt{Value} of the input at the given \texttt{Index}& ns=4;i=1066& ns=4;i=1067\\
		\hline
		WriteOutput(Index, Value) & Sets the output at the given \texttt{Index} to the given \texttt{Value}& ns=4;i=1066 & ns=4;i=1111\\
		\hline
		GetMostRecentError() & Returns the most recent error that occurred & ns=4;i=1103 & ns=4;i=1105\\
		\hline
		GetRecentErrors(NumberOfErrors) & Returns the last \texttt{NumberOfErrors} errors that occurred& ns=4;i=1066 & ns=4;i=1140\\
		\hline
		ResetError() & Resets the error state of the robot controller& ns=4;i=1102 & ns=4;i=1108\\
		\hline
		OpenGripper() & Opens the multi function gripper& ns=4;i=1130 & ns=4;i=1131\\
		\hline
		CloseGripper() & Closes the multi function gripper& ns=4;i=1130& ns=4;i=1134\\
		\hline
		Move(X,Y,Z,A,B,C) & Translates the robot arm to the position (\texttt{X,Y,Z}) with the rotation (\texttt{A,B,C})& ns=4;i=1130 & ns=4;i=1137\\
		\hline
		MoveToSafePosition() & Moves the robot arm into the predefined safety position & ns=4;i=1113&ns=4;i=1115\\
		\hline
		RestartServer() & Reboots the OPC UA Gateway device & ns=0;i=2253& ns=4;i=1164\\
		\hline
	\end{tabularx}
	\caption{Implemented OPC UA methods}
	\label{tab:methods}
\end{table}
The I/O mapping is needed for the first two methods, ReadInput and WriteOutput where the needed index parameter corresponds to the indices given in \autoref{tab:io}.\newline
The ObjectID and the MethodId can directly be used in the NodeRED interface implementation in this laboratory.
\newpage
\section{OPC UA Objects}
The OPC UA methods from \autoref{tab:methods} are encapsulated in the OPC UA objects given in \autoref{tab:objects}.
\begin{table}[H]
	\setlength\extrarowheight{4pt}
	\small
	\begin{tabularx}{\textwidth}{|p{4cm}|X|p{2cm}|}
		\hline
		\multicolumn{3}{|c|}{\bf \color{black} \large OPC UA Objects}\\
		\hline\hline
		\bf Object &\bf Description & \bf ObjectId \\
		\hline\hline
		MethodSet (Controller) & The MethodSet object, which is a component of the Controller object & ns=4;i=1066 \\ \hline
		GetMostRecentError & The GetMostRecentError object, which is a component of the ErrorLog Functions folder & ns=4;i=1103 \\ \hline
		ResetError & The ResetError object, which is a component of the ResetError Functions folder & ns=4;i=1102 \\ \hline
		MethodSet (MotionDevice) & The MethodSet object, which is a component of the MotionDevice object & ns=4;i=1130 \\ \hline
		MoveToSafePosition & The MoveToSafePosition object, which is a component of the MoveToSafeState Functions folder & ns=4;i=1113 \\ \hline
		Server & The Server object, which is organized by the Objects folder & ns=0;i=2253 \\ \hline
	\end{tabularx}
	\caption{OPC UA objects}
	\label{tab:objects}
\end{table}

\section{OPC UA Namespaces}
As it can be seen in \autoref{tab:names}, most of the objects and therefore methods are provided by the namespace 4, Mitsubishi Electric Robot Nodeset, with the exception of the restartServer() method in the Server object (ns=0). This method is capsuled from the Robot namespace as it will actually restart the OPC UA gateway Raspberry Pi and therefore acts on the Server directly.
\begin{table}[H]
	\setlength\extrarowheight{4pt}
	\small
	\begin{tabularx}{\textwidth}{|p{3cm}|p{8cm}|X|}
		\hline
		\multicolumn{3}{|c|}{\bf \color{black} \large OPC UA Namespaces}\\
		\hline\hline
		\bf NamespaceIndex &\bf NamespaceUri & \bf Description \\
		\hline\hline
		0 & http://opcfoundation.org/UA/ & Standard Nodeset\\ \hline
		2 & http://opcfoundation.org/UA/DI/ & Devices Nodeset\\ \hline
		3 & http://opcfoundation.org/UA/Robotics/ & Robotics Nodeset\\ \hline
		4 & http://auto.tuwien.ac.at/UA/MitsubishiElectricRobot/ & Mitsubishi Electric Robot Nodeset\\ \hline
	\end{tabularx}
	\caption{OPC UA namespaces}
	\label{tab:names}
\end{table}

\end{document}
